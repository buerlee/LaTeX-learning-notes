\documentclass{ctexart}
\usepackage{amsmath,ctex,caption,multirow,hyperref,newfloat,tabularx,makecell}
\usepackage[a4paper,left=1cm,right=1cm,top=0cm,bottom=0cm]{geometry}
\renewcommand{\rm}{\,\mathrm}
\DeclareFloatingEnvironment[within=section]{表}
\hypersetup{
	bookmarks=true,
	bookmarksopen=true
}

\title{模电数电实验数据}
\date{\today}
\author{死抠}
\begin{document}
	\maketitle
	\tt{数据仅作学习参考用途,用作实验报告后果自负。}
	\section{模拟电子技术实验数据}
	\subsection{模电实验一\quad 常用电子仪器使用练习}
	\begin{table}[h]
		\centering
		\caption{万用表、示波器测量电压实验数据}
		\begin{tabular}{|c|c|c|c|c|c|}
			\hline
			稳压源表头指示 & $6\rm{V}$ & $12\rm{V}$ & $18\rm{V}$ & $24\rm{V}$ & $30\rm{V}$\\
			\hline
			万用表读数 & $6.03$ & $12.03$ & $18.02$ & $24.00$ & $30.00$\\
			\hline
			示波器读数 & & & & &\\
			\hline
		\end{tabular}
	\end{table}
    \begin{table}[h]
    	\centering
    	\caption{示波器测量电压实验数据}
    	\begin{tabular}{|c|c|c|c|c|}
    		\hline
    		校正信号 & 标称值 & \multicolumn{2}{|c|}{示波器测得的原始数据} & 测量值\\
    		\hline
    		幅度 $U_{P-P}$ & $2\rm{V}$ & $3.5\rm{div}$ & $0.5\rm{V/div}$ & $1.75\rm{V}$\\
    		\hline
    		频率 $f$ & $1000\rm{Hz}$ & $4\rm{div}$ & $0.25\rm{ms/div}$ & $1\rm{Hz}$\\
    		\hline
    	\end{tabular}
    \end{table}
    \begin{table}[h]
    	\centering
    	\caption{交流电压测量}
    	\begin{tabular}{|c|c|c|c|c|c|}
    		\hline
    		\multirow{2}*{信号电压频率} & \multicolumn{2}{|c|}{示波器测量值} & \multirow{2}*{信号电压毫伏表读数$/\rm{V}$} & \multicolumn{2}{|c|}{示波器测量值}\\
    		 \cline{2-3}\cline{5-6}
    		 & 周期$/\rm{ms}$ & 频率$/\rm{Hz}$ & & 峰峰值$/\rm{V}$ & 有效值$/\rm{V}$\\
    		 \hline
    		 $50\rm{Hz}$ & $20$ & $50.00$ & $0.896$ & $2.58$ & $0.912$\\
    		 \hline
    		 $100\rm{Hz}$ & $10$ & $100.00$ & $0.900$ & $2.48$ & $0.877$\\
    		 \hline
    		 $1\rm{kHz}$ & $1$ & $1.00\times10^3$ & $0.893$ & $2.50$ & $0.905$\\
    		 \hline
    		 $10\rm{kHz}$ & $99.92\times10^{-3}$ & $10.01\times10^3$ & $0.571$ & $2.56$ & $0.905$\\
    		 \hline
    	\end{tabular}
    \end{table}

    \subsection{模电实验二\quad 共射单管交流放大电路}
    \begin{table}[h]
    	\centering
    	\caption{静态工作点测试表}
    	\begin{tabular}{|c|c|c|c|c|c|c|}
    		\hline
    		\multicolumn{4}{|c|}{测量值} & \multicolumn{3}{|c|}{计算值}\\
    		\hline
    		$U_{\rm{B}}$ & $U_{\rm{C}}$ & $U_{\rm{E}}$ & $R_{\rm{b}}$ & $U_{\rm{BE}}$ & $U_{\rm{CE}}$ & $I_{\rm{CQ}}$\\
    		\hline
    		$0.642$ & $5.07$ & $0$ & $687$ & & &\\
    		 \hline
    	\end{tabular}
    \end{table}
    \begin{table}[h]
    	\centering
    	\caption{电压放大倍数测试表}
    	\begin{tabular}{|c|c|c|}
    		\hline
    		\multicolumn{2}{|c|}{测量值} & 计算值\\
    		\hline
    		$V_{\rm{i}}/\rm{mV}$ & $V_{\rm{o}}/\rm{V}$ & $A_{\rm{V}}$\\
    		\hline
    		$12$ & $2.618$ & \\
    		\hline
    		$14$ & $2.988$ & \\
    		\hline
    		$16$ & $3.268$ & \\
    		\hline
    		$20$ & $3.671$ & \\
    		\hline
    	\end{tabular}
    \end{table}
    \begin{table}[h]
    	\centering
    	\caption{电路参数变化对电压放大倍数及输出波形的影响}
    	\begin{tabular}{|c|c|c|c|c|c|c|c|c|c|}
    		\hline
    		\multicolumn{3}{|c|}{给定参数} & \multicolumn{3}{|c|}{测量结果} & \multicolumn{3}{|c|}{由测量值计算} & \multirow{2}*{波形失真类型}\\
    		\cline{1-9}
    		$R_{\rm{b}}$ & $R_{\rm{C}}/\rm{k}\Omega$ & $R_{\rm{L}}/\rm{k}\Omega$ & $U_{\rm{CE}}$ & $U_{\rm{O}}$ & 输出波形图 & $I_{\rm{CQ}}$ & $I_{\rm{BQ}}$ & $A_{\rm{U}}$ &\\
    		\hline
    		\multirow{3}*{合适} & $3.9$ & $\infty$ & $0.439$ & $3.426$ & 略 & & & & \\
    		\cline{2-10}
    		 & $2$ & $2.2$ & $1.800$ & $1.779$ & 略 & & & & \\
    		\cline{2-10}
    		 & $2$ & $\infty$ & $3.226$ & $3.241$ & 略 & & & & \\
    		\hline
    		变小 & \multicolumn{2}{|c|}{$R_{\rm{C}}=2\rm{k}\Omega$} & $2.732$ & $2.320$ & 略 & & & & \\
    		\cline{1-1}\cline{4-10}
    		变大 & \multicolumn{2}{|c|}{$R_{\rm{L}}=\infty$} & $5.26$ & $3.20$ & 略 & & & & \\
    		\hline
    	\end{tabular}
    \end{table}
    \subsection{模电实验四\quad 差动放大电路}
    \begin{table}[ht]
    	\centering
    	\caption{静态工作点记录表}
    	\begin{tabular}{|c|c|c|c|c|c|c|c|}
    		\hline
    		\multirow{2}*{测量值} & $U_{\rm{c}1}/\rm{V}$ & $U_{\rm{B}1}/\rm{V}$ & $U_{\rm{E}1}/\rm{V}$ & $U_{\rm{c}2}/\rm{V}$ & $U_{\rm{B}2}/\rm{V}$ & $U_{\rm{E}2}/\rm{V}$ & $U_{\rm{R}_{e1}}/\rm{V}$\\
    		\cline{2-8}
    		 & $6.98$ & $0$ & $-0.609$ & $7.03$ & $0$ & $-0.607$ & $11.24$\\
    		 \hline
    	\end{tabular}
    \end{table}
    \begin{table}[ht]
    	\centering
    	\caption{电压放大倍数测量记录表}
    	\begin{tabular}{*{14}{|c}}
    		\hline
    		\multirowcell{3}{测量及计\\算值输入\\信号 $/\rm{V}$} & \multicolumn{6}{|c|}{差模输入} & \multicolumn{6}{|c|}{共模输入} & 共模抑制比\\
    		\cline{2-14}
    		 & \multicolumn{3}{|c|}{测量值} & \multicolumn{3}{|c|}{计算值} & \multicolumn{3}{|c|}{测量值} & \multicolumn{3}{|c|}{计算值} & 计算值\\
    		 \cline{2-14}
    		 & $V_{\rm{c}1}$ & $V_{\rm{c}2}$ & $V_{\rm{o}\text{双}}$ & $A_{\rm{d}1}$ & $A_{\rm{d}2}$ & $A_{\text{双}}$ & $V_{\rm{c}1}$ & $V_{\rm{c}2}$ & $V_{\rm{o}\text{双}}$ & $V_{\rm{c}1}$  & $V_{\rm{c}2}$ & $V_{\text{双}}$ & $KCMR=|A_{\rm{d}}/A_{\rm{c}}|$\\
    		 \hline
    		 \multirowcell{2}{$V_{\rm{i}1}=+0.04$ \\ $V_{\rm{i}2}=-0.04$} & $6.57$ & $7.40$ & $-0.83$ &  &  &  &  $6.96$ & $6.97$ & $-0.004$ & & & &  \\
    		 \cline{2-14}
    		  & & & & & & & & & & & & & \\
    		  \hline
    		  \multirowcell{2}{$V_{\rm{i}1}=+0.12$\\$V_{\rm{i}2}=-0.12$} & $5.73$ & $8.14$ & $-2.394$ &  &  &  & $6.88$ & $6.92$ & $-0.001$ &  &  &  &  \\
    		  \cline{2-14}
    		  & & & & & & & & & & & & & \\
    		  \hline
    		  \multirowcell{2}{$V_{\rm{i}1}=+0.2$\\$V_{\rm{i}2}=-0.2$} & $5.09$ & $8.56$ & $-3.512$ & & & & $6.79$ & $6.83$ & $-0.0256$ & & & & \\
    		  \cline{2-14}
    		  & & & & & & & & & & & & & \\
    		  \hline
    	\end{tabular}
    \end{table}
    \begin{table}[htbp]
    	\centering
    	\caption{单端输入的差动电路的测量记录表}
    	\begin{tabular}{rrrrrrr}
    		\hline
    		\multicolumn{1}{|c|}{\multirow{3}{*}{输入信号}} & \multicolumn{6}{c|}{单端输入} \\
    		\cline{2-7}    \multicolumn{1}{|c|}{} & \multicolumn{3}{c|}{测量值} & \multicolumn{3}{c|}{计算值} \\
    		\cline{2-7}    \multicolumn{1}{|c|}{} & \multicolumn{1}{c|}{$V_{\rm{c}1}/\rm{V}$} & \multicolumn{1}{c|}{$V_{\rm{c}2}/\rm{V}$} & \multicolumn{1}{c|}{$V_{\rm{o}}/\rm{V}$} & \multicolumn{1}{c|}{$A_{\rm{c}1}$} & \multicolumn{1}{c|}{$A_{\rm{c}2}$} & \multicolumn{1}{c|}{$A_{\text{双}}$} \\
    		\hline
    		\multicolumn{1}{|c|}{直流$+0.2\rm{V}$} & \multicolumn{1}{c|}{$2.059$} & \multicolumn{1}{c|}{$9.93$} & \multicolumn{1}{c|}{$-7.94$} & \multicolumn{1}{c|}{} & \multicolumn{1}{c|}{} & \multicolumn{1}{c|}{} \\
    		\hline
    		\multicolumn{1}{|c|}{直流$-0.2\rm{V}$} & \multicolumn{1}{c|}{$5.07$} & \multicolumn{1}{c|}{$3.95$} & \multicolumn{1}{c|}{$1.037$} & \multicolumn{1}{r|}{} & \multicolumn{1}{r|}{} & \multicolumn{1}{r|}{} \\
    		\hline
    		\multicolumn{1}{|c|}{正弦波交流信号} & \multicolumn{1}{r|}{} & \multicolumn{1}{r|}{} & \multicolumn{1}{r|}{} & \multicolumn{1}{r|}{} & \multicolumn{1}{r|}{} & \multicolumn{1}{r|}{} \\
    		\hline
    	\end{tabular}%
    \end{table}%
    \subsection{负反馈放大电路}
    \begin{table}[htbp]
    	\centering
    	\caption{静态工作点}
    	\begin{tabular}{|l|c|c|c|}
    		\hline
    		& \multicolumn{1}{c|}{$U_{\rm{B}}/V$} & \multicolumn{1}{c|}{$U_{\rm{E}}/V$} & \multicolumn{1}{c|}{$U_{\rm{c}}/V$} \\
    		\hline
    		第一级   & $6.85$  & $6.2$   & $6.26$ \\
    		\hline
    		第二级   & $3.222$ & $2.606$ & $9.58$ \\
    		\hline
    	\end{tabular}%
    \end{table}%
    \begin{table}[htbp]
    	\centering
    	\caption{放大倍数的测量}
    	\begin{tabular}{|c|c|c|c|r|}
    		\hline
    		& \multicolumn{1}{c|}{$R_{\rm{L}}$} & \multicolumn{1}{c|}{$V_{\rm{i}}/\rm{mV}$} & \multicolumn{1}{c|}{$V_{\rm{o}}/\rm{mV}$} & \multicolumn{1}{c|}{$A_{\rm{v}}=V_{\rm{o}}/V_{\rm{i}}$} \\
    		\hline
    		\multirow{2}{*}{开环} &   $\infty$    & $10$    & $106.06$ &  \\
    		\cline{2-5}          & $4.7\rm{k}$  & $10$    & $76.37$ &  \\
    		\hline
    		\multirow{2}{*}{闭环} &   $\infty$    & $10$    & $46$    &  \\
    		\cline{2-5}          & $4.7\rm{k}$  & $10$    & $40$    &  \\
    		\hline
    	\end{tabular}%
    \end{table}%
    \begin{table}[htbp]
    	\centering
    	\caption{输入电阻的测量}
    	\begin{tabular}{|l|c|c|c|}
    		\hline
    		& \multicolumn{1}{c|}{$V_{\rm{s}}/\rm{V}$} & \multicolumn{1}{c|}{$V_{\rm{i}}/\rm{V}$} & \multicolumn{1}{c|}{$R_{\rm{i}}$} \\
    		\hline
    		开环    & $0.02$  & $0.01$  &  \\
    		\hline
    		闭环    & $0.02$  & $0.009$ &  \\
    		\hline
    	\end{tabular}%
    \end{table}%
    \begin{table}[htbp]
    	\centering
    	\caption{输出电阻的测量}
    	\begin{tabular}{|c|c|c|c|}
    		\hline
    		& $V_{\rm{o}}(R_{\rm{L}}=\infty)$     & $V_{\rm{o}_{\rm{L}}}(R_{\rm{L}}=4.7\rm{k})$     & \multicolumn{1}{c|}{$R_{\rm{o}}$} \\
    		\hline
    		开环    & 0.106 & 0.076 &  \\
    		\hline
    		闭环    & 0.046 & 0.04  &  \\
    		\hline
    	\end{tabular}%
    \end{table}%
    % Table generated by Excel2LaTeX from sheet 'Sheet1'
    \begin{table}[htbp]
    	\centering
    	\caption{负反馈对频率特性的影响}
    	\begin{tabular}{|c|c|c|c|}
    		\hline
    		\multirow{2}{*}{基本放大器} & $f_{\rm{L}}/\rm{Hz}$     & $f_{\rm{H}}/\rm{kHz}$     & $\triangle f/\rm{kHz}$ \\
    		\cline{2-4}          & 35    & 8     &  \\
    		\hline
    		\multirow{2}{*}{负反馈放大器} & $f_{\rm{Lf}}/\rm{Hz}$     & $f_{\rm{Hf}}/\rm{kHz}$     & $\triangle f_{\rm{f}}/\rm{kHz}$ \\
    		\cline{2-4}          & 17    & 6     &  \\
    		\hline
    	\end{tabular}%
    \end{table}%
    \newpage
    \section{数字电子技术实验数据}
    \subsection{数电实验一\quad 门电路}
    \begin{table}[h]
    	\centering
    	\caption{$\rm{TTL}$ 基本门电路测试表}
    	\begin{tabular}{|c|c|c|c|c|}
    		\hline
    		\multicolumn{2}{|c}{输入} & \multicolumn{3}{|c|}{输出}\\
    		\hline
    		$A$ & $B$ & 与非门$Y$ & 或门$Y$ & 异或门$Y$\\
    		\hline
    		$0$ & $0$ & $1$ & $0$ & $0$\\
    		\hline
    		$0$ & $1$ & $1$ & $1$ & $1$\\
    		\hline
    		$1$ & $0$ & $1$ & $1$ & $1$\\
    		\hline
    		$1$ & $1$ & $0$ & $1$ & $0$\\
    		\hline
    	\end{tabular}
    \end{table}
    \begin{table}[h]
    	\centering
    	\caption{$\rm{TTL}$四输入端双与非门测试表}
    	\begin{tabular}{|c|c|c|c|c|c|}
    		\hline
    		\multicolumn{4}{|c|}{输入} & \multicolumn{2}{|c|}{输出}\\
    		\hline
    		$A$ & $B$ & $C$ & $D$ & $Y$ & 电压$/\rm{V}$\\
    		\hline
    		$1$ & $1$ & $1$ & $1$ & $0$ & $0.03$\\
    		\hline
    		$0$ & $1$ & $1$ & $1$ & $1$ & $4.40$\\
    		\hline
    		$0$ & $0$ & $1$ & $1$ & $1$ & $4.40$\\
    		\hline
    		$0$ & $0$ & $0$ & $1$ & $1$ & $4.41$\\
    		\hline
    		$0$ & $0$ & $0$ & $0$ & $1$ & $4.40$\\
    		\hline
    	\end{tabular}
    \end{table}
    \begin{table}[h]
    	\centering
    	\caption{$\rm{TTL}$与非门的电压传输特性测试表}
    	\begin{tabular}{|c|c|c|c|c|c|c|c|c|}
    		\hline
    		$V_{\mathrm{i}}/\rm{V}$ & $0$ & $1$ & $2$ & $2.5$ & $2.6$ & $2.8$ & $2.9$ & $3$\\
    		\hline
    		$V_{\mathrm{o}}/\rm{V}$ & $4.85$ & $4.85$ & $4.85$ & $2.30$ & $0.02$ & $0.02$ & $0.02$ & $0.02$\\
    		\hline
    	\end{tabular}
    \end{table}

    \subsection{数电实验二\quad 译码器}
    \begin{table}[h]
    	\centering
    	\caption{$74\rm{LS}138$的逻辑功能测试表}
    	\begin{tabular}{|c|c|c|c|c|c|c|c|c|c|c|c|c|}
    		\hline
    		\multicolumn{5}{|c|}{输入} & \multicolumn{8}{|c|}{\multirow{2}*{输出}}\\
    		\cline{1-5}
    		\multicolumn{2}{|c|}{允许} & \multicolumn{3}{|c|}{选择} & \multicolumn{8}{|c|}{}\\
    		\hline
    		$S1$ & $S2'+S3'$ & $A2$ & $A1$ & $A0$ & $Y0'$ & $Y1'$ & $Y2'$ & $Y3'$ & $Y4'$ & $Y5'$ & $Y6'$ & $Y7'$\\
    		\hline
    		X & 1 & X & X & X & $1$ & $1$ & $1$ & $1$ & $1$ & $1$ & $1$ & $1$\\
    		\hline
    		0 & X & X & X & X & $1$ & $1$ & $1$ & $1$ & $1$ & $1$ & $1$ & $1$\\
    		\hline
    		1 & 0 & 0 & 0 & 0 & $0$ & $1$ & $1$ & $1$ & $1$ & $1$ & $1$ & $1$\\
    		\hline
    		1 & 0 & 0 & 0 & 1 & $1$ & $0$ & $1$ & $1$ & $1$ & $1$ & $1$ & $1$\\
    		\hline
    		1 & 0 & 0 & 1 & 0 & $1$ & $1$ & $0$ & $1$ & $1$ & $1$ & $1$ & $1$\\
    		\hline
    		1 & 0 & 0 & 1 & 1 & $1$ & $1$ & $1$ & $0$ & $1$ & $1$ & $1$ & $1$\\
    		\hline
    		1 & 0 & 1 & 0 & 0 & $1$ & $1$ & $1$ & $1$ & $0$ & $1$ & $1$ & $1$\\
    		\hline
    		1 & 0 & 1 & 0 & 1 & $1$ & $1$ & $1$ & $1$ & $1$ & $0$ & $1$ & $1$\\
    		\hline
    		1 & 0 & 1 & 1 & 0 & $1$ & $1$ & $1$ & $1$ & $1$ & $1$ & $0$ & $1$\\
    		\hline
    		1 & 0 & 1 & 1 & 1 & $1$ & $1$ & $1$ & $1$ & $1$ & $1$ & $1$ & $0$\\
    		\hline
    	\end{tabular}
    \end{table}
    \begin{table}
    	\centering
    	\caption{74LS48逻辑功能测试表}
    	\begin{tabular}{|c|c|c|c|c|c|c|c|c|c|c|}
    		\hline
    		\multicolumn{4}{|c|}{输入} & \multicolumn{7}{|c|}{输出}\\
    		\hline
    		$D3$ & $D2$ & $D1$ & $D0$ & $a$ & $b$ & $c$ & $d$ & $e$ & $f$ & $g$\\
    		\hline
    		$0$ & $0$ & $0$ & $0$ & $1$ & $1$ & $1$ & $1$ & $1$ & $1$ & $0$\\
    		\hline
    		$0$ & $0$ & $0$ & $1$ & $0$ & $1$ & $1$ & $0$ & $0$ & $0$ & $0$\\
    		\hline
    		$0$ & $0$ & $1$ & $0$ & $1$ & $1$ & $0$ & $1$ & $1$ & $0$ & $1$\\
    		\hline
    		$0$ & $0$ & $1$ & $1$ & $1$ & $1$ & $1$ & $1$ & $0$ & $0$ & $1$\\
    		\hline
    		$0$ & $1$ & $0$ & $0$ & $0$ & $1$ & $1$ & $0$ & $0$ & $1$ & $1$\\
    		\hline
    		$0$ & $1$ & $0$ & $1$ & $1$ & $0$ & $1$ & $1$ & $0$ & $1$ & $1$\\
    		\hline
    		$0$ & $1$ & $1$ & $0$ & $0$ & $0$ & $1$ & $1$ & $1$ & $1$ & $1$\\
    		\hline
    		$0$ & $1$ & $1$ & $1$ & $1$ & $1$ & $1$ & $0$ & $0$ & $0$ & $0$\\
    		\hline
    		$1$ & $0$ & $0$ & $0$ & $1$ & $1$ & $1$ & $1$ & $1$ & $1$ & $1$\\
    		\hline
    		$1$ & $0$ & $0$ & $1$ & $1$ & $1$ & $1$ & $0$ & $0$ & $1$ & $1$\\
    		\hline
    	\end{tabular}
    \end{table}
\end{document}
