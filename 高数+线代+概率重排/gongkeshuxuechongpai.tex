\documentclass[cn,11pt,fancy,hide]{elegantbook}
\setlist[enumerate]{leftmargin=0pt}

\title{工科数学试卷汇总}
\subtitle{高数、线代、概率、复变}

\author{sikouhjw、xajzh}
\institute{临时组织起来的重排小组}
\date{\today}
\version{1.00}

\equote{“不论一个人的数学水平有多高,只要对数学拥有一颗真诚的心,他就在自己的心灵上得到了升华。”---SCIbird}

\logo{logo}
\cover{cover.jpg}
\newcommand{\ee}{\mathrm{e}}
\newcommand{\dd}{\,\mathrm{d}}
\newcommand{\ii}{\mathrm{i}}
\newcommand{\Ln}{\mathrm{Ln\,}}
\renewcommand{\leq}{\leqslant}
\everymath{\displaystyle}
\newcommand{\fourch}[4]{\\\begin{tabular}{*{4}{@{}p{3.5cm}}}(A)~#1 & (B)~#2 & (C)~#3 & (D)~#4\end{tabular}} % 一行
\newcommand{\twoch}[4]{\\\begin{tabular}{*{2}{@{}p{7cm}}}(A)~#1 & (B)~#2\end{tabular}\\\begin{tabular}{*{2}{@{}p{7cm}}}(C)~#3 & (D)~#4\end{tabular}}  %两行
\newcommand{\onech}[4]{\\(A)~#1 \\ (B)~#2 \\ (C)~#3 \\ (D)~#4}  % 四行
\begin{document}
\maketitle
\tableofcontents

% \thispagestyle{empty}

\mainmatter
\hypersetup{pageanchor=true}
\chapter{声明}
本汇总不得用于商业用途,最新版下载地址:\href{https://github.com/sikouhjw/LaTeX-learning-notes/tree/master/%E9%AB%98%E6%95%B0+%E7%BA%BF%E4%BB%A3+%E6%A6%82%E7%8E%87%E9%87%8D%E6%8E%92}{Github},不保证题目、答案的正确性,如有错误可通过QQ群\footnote{991832226}或者邮箱\footnote{489765924@qq.com}联系我们

\chapter{高等数学试卷汇总}

\section{高数(一)期中}

\subsection{2018-2019A7}
\subsubsection{选择题}
\begin{enumerate}
	\item 微分方程 $(y')^3+3\sqrt{y''}+x^4y'''=\sin x$ 的阶数是(\hspace{1pc})
	\fourch{1}{4}{2}{3}
	\item 设 $f(x,y)=x-y-\sqrt{x^2+y^2}$ , 则 $f_{x}(3,4)=$(\hspace{1pc})
	\fourch{$\frac{3}{5}$}{$\frac{2}{5}$}{$-\frac{2}{5}$}{$\frac{1}{5}$}
	\item 微分方程 $y'=\frac{y}{x}$ 的一个特解是(\hspace{1pc})
	\fourch{$y=2x$}{$\ee^y=x$}{$y=x^2$}{$y=\ln x$}
	\item 若 $z=\ln\sqrt{1+x^2+y^2}$ , 则 $\left.\dd z\right|_{(1,1)}=$(\hspace{1pc})
	\fourch{$\frac{\dd x+\dd y}{3}$}{$\frac{\dd x+\dd y}{2}$}{$\frac{\dd x+\dd y}{1}$}{$3(\dd x+\dd y)$}
	\item 设直线 $L:\begin{cases}
	x+3y+2z+1=0\\
	2x-y-10z+3=0
	\end{cases}$ , 平面 $\eta:\ 4x-2y+z-2=0$ , 则(\hspace{1pc})
	\fourch{$L$ 在 $\eta$ 上}{$L$ 平行于 $\eta$}{$L$ 垂直于 $\eta$}{$L$ 与 $\eta$ 斜交}
	\item 方程 $y'+3xy=6x^2y$ 是(\hspace{1pc})
	\twoch{二阶微分方程}{非线性微分方程}{一阶线性非齐次微分方程}{可分离变量的微分方程}
	\item 曲面 $\frac{x^2}{9}-\frac{y^2}{4}+\frac{z^2}{4}=1$ 与平面 $x=y$ 的交线是(\hspace{1pc})
	\fourch{两条直线}{双曲线}{椭圆}{抛物线}
	\item 设 $z=\ee^{x^2y}$ , 则 $\frac{\partial^2z}{\partial x\partial y}=$(\hspace{1pc})
	\twoch{$2y\left(1+x^3\right)\ee^{x^2y}$}{$\ee^{x^2y}$}{$2x\left(1+x^2y\right)\ee^{x^2y}$}{$2x\ee^{x^2y}$}
	\item 下列结论正确的是(\hspace{1pc})
	\twoch{$\vec{a}\times\left(\vec{b}-\vec{c}\right)=\vec{a}\times\vec{b}-\vec{a}\times\vec{c}$}{若 $\vec{a}\times\vec{b}=\vec{a}\times\vec{c}$ 且 $\vec{a}\ne\vec{0}$ , 则 $\vec{b}=\vec{c}$}{$\vec{a}\times\vec{b}=\vec{b}\times\vec{a}$}{若 $\left|\vec{a}\right|=1,\left|\vec{b}\right|=1$ , 则 $\left|\vec{a}\times\vec{b}\right|=1$}
\end{enumerate}

\subsubsection{填空题}
\begin{enumerate}
	\item 平面过点 $(2,0,0),(0,1,0),(0,0,0.5)$ , 则该平面的方程是\underline{\hspace{8pc}}
	\item 设 $y_1$ 是 $y''+p(x)y'+q(x)y=f(x)$ 的解, $y_2$ 是 $y''+p(x)y'+q(x)y=f(x)$ 的解, 则 $y_1+y_2$ 是\underline{\hspace{8pc}}方程的解
	\item 设 $z=y\arctan x$ , 则 $\left.\mathrm{grad}\,z\right|_{(1,2)}=$\underline{\hspace{8pc}}
	\item 过点 $P(0,2,4)$ 且与两平面 $x+2z=1$ 和 $y-2z=2$ 平行的直线方程是\underline{\hspace{8pc}}
	\item 设 $f(x,y)=\arcsin\frac{y}{x}$ , 则 $f_y(1,0)=$\underline{\hspace{8pc}}
	\item $y=\ee^x$ 是微分方程 $y''+py'+6y=0$ 的一个特解, 则 $p=$\underline{\hspace{8pc}}
	\item 已知平面 $\eta_1:\ A_1x+B_1y+C_1z+D_1=0$ 与平面 $\eta_2:\ A_2x+B_2y+C_2z+D_2=0$, 则 $\eta_1\perp\eta_2$ 的充要条件是\underline{\hspace{8pc}}
	\item 微分方程 $y''+2y'+5y=0$ 的通解为 $y=$\underline{\hspace{8pc}}
	\item 设 $z=\ee^{xy}+\cos\left(x^2+y\right)$, 则 $\frac{\partial z}{\partial y}=$\underline{\hspace{8pc}}
\end{enumerate}
\subsubsection{大题}
\begin{enumerate}
	\item 求方程 $\frac{\dd z}{\dd x}=-z+4x$ 的通解
	\item 求曲线 $2z+1=\ln(xy)+\ee^z$ 在点 $M_{0}(1,1,0)$ 处的切平面和法线方程
	\item 设由方程组 $\begin{cases}
	x+y+z=0\\
	x^2+y^2+z^2=1
	\end{cases}$
	确定了隐函数 $x=x(z),y=y(z)$ , 求 $\frac{\dd x}{\dd z},\frac{\dd y}{\dd z}$
	\item 求方程 $y''+6y'+13y=\ee^t$ 的通解
	\item 设 $z=x^2y+\sin x+\varphi(xy+1)$ , 且 $\varphi(u)$ 具有一阶连续导数, 求 $\frac{\partial z}{\partial x},\frac{\partial z}{\partial y}$
\end{enumerate}

\subsection{2018-2019A7答案}

\subsubsection{选择题}
\begin{enumerate}
	\item 微分方程 $(y')^3+3\sqrt{y''}+x^4y'''=\sin x$ 的阶数是(\hspace{0.25pc}D\hspace{0.25pc})
	\fourch{1}{4}{2}{3}
	\item 设 $f(x,y)=x-y-\sqrt{x^2+y^2}$ , 则 $f_{x}(3,4)=$(\hspace{0.25pc}B\hspace{0.25pc})
	\fourch{$\frac{3}{5}$}{$\frac{2}{5}$}{$-\frac{2}{5}$}{$\frac{1}{5}$}
	\item 微分方程 $y'=\frac{y}{x}$ 的一个特解是(\hspace{0.25pc}A\hspace{0.25pc})
	\fourch{$y=2x$}{$\ee^y=x$}{$y=x^2$}{$y=\ln x$}
	\item 若 $z=\ln\sqrt{1+x^2+y^2}$ , 则 $\left.\dd z\right|_{(1,1)}=$(\hspace{0.25pc}A\hspace{0.25pc})
	\fourch{$\frac{\dd x+\dd y}{3}$}{$\frac{\dd x+\dd y}{2}$}{$\frac{\dd x+\dd y}{1}$}{$3(\dd x+\dd y)$}
	\item 设直线 $L:\begin{cases}
	x+3y+2z+1=0\\
	2x-y-10z+3=0
	\end{cases}$ , 平面 $\eta:\ 4x-2y+z-2=0$ , 则(\hspace{0.25pc}C\hspace{0.25pc})
	\fourch{$L$ 在 $\eta$ 上}{$L$ 平行于 $\eta$}{$L$ 垂直于 $\eta$}{$L$ 与 $\eta$ 斜交}
	\item 方程 $y'+3xy=6x^2y$ 是(\hspace{0.25pc}D\hspace{0.25pc})
	\twoch{二阶微分方程}{非线性微分方程}{一阶线性非齐次微分方程}{可分离变量的微分方程}
	\item 曲面 $\frac{x^2}{9}-\frac{y^2}{4}+\frac{z^2}{4}=1$ 与平面 $x=y$ 的交线是(\hspace{0.25pc}B\hspace{0.25pc})
	\fourch{两条直线}{双曲线}{椭圆}{抛物线}
	\item 设 $z=\ee^{x^2y}$ , 则 $\frac{\partial^2z}{\partial x\partial y}=$(\hspace{0.25pc}C\hspace{0.25pc})
	\twoch{$2y\left(1+x^3\right)\ee^{x^2y}$}{$\ee^{x^2y}$}{$2x\left(1+x^2y\right)\ee^{x^2y}$}{$2x\ee^{x^2y}$}
	\item 下列结论正确的是(\hspace{0.25pc}A\hspace{0.25pc})
	\twoch{$\vec{a}\times\left(\vec{b}-\vec{c}\right)=\vec{a}\times\vec{b}-\vec{a}\times\vec{c}$}{若 $\vec{a}\times\vec{b}=\vec{a}\times\vec{c}$ 且 $\vec{a}\ne\vec{0}$ , 则 $\vec{b}=\vec{c}$}{$\vec{a}\times\vec{b}=\vec{b}\times\vec{a}$}{若 $\left|\vec{a}\right|=1,\left|\vec{b}\right|=1$ , 则 $\left|\vec{a}\times\vec{b}\right|=1$}
\end{enumerate}

\subsubsection{填空题}
\begin{enumerate}
	\item 平面过点 $(2,0,0),(0,1,0),(0,0,0.5)$ , 则该平面的方程是\underline{\hspace{1pc}$\frac{x}{2}+y+2z=1$\hspace{1pc}}
	\item 设 $y_1$ 是 $y''+p(x)y'+q(x)y=f(x)$ 的解, $y_2$ 是 $y''+p(x)y'+q(x)y=f(x)$ 的解, 则 $y_1+y_2$ 是\underline{\hspace{1pc}$y''+p(x)y'+q(x)y=2f(x)$\hspace{1pc}}方程的解
	\item 设 $z=y\arctan x$ , 则 $\left.\mathrm{grad}\,z\right|_{(1,2)}=$\underline{\hspace{1pc}$\dd x+\frac{\uppi}{4}\dd y$\hspace{1pc}}
	\item 过点 $P(0,2,4)$ 且与两平面 $x+2z=1$ 和 $y-2z=2$ 平行的直线方程是\underline{\hspace{1pc}$\frac{x}{-2}=\frac{y-2}{2}=\frac{z-4}{1}$\hspace{1pc}}
	\item 设 $f(x,y)=\arcsin\frac{y}{x}$ , 则 $f_y(1,0)=$\underline{\hspace{1pc}$1$\hspace{1pc}}
	\item $y=\ee^x$ 是微分方程 $y''+py'+6y=0$ 的一个特解, 则 $p=$\underline{\hspace{1pc}$-7$\hspace{1pc}}
	\item 已知平面 $\eta_1:\ A_1x+B_1y+C_1z+D_1=0$ 与平面 $\eta_2:\ A_2x+B_2y+C_2z+D_2=0$, 则 $\eta_1\perp\eta_2$ 的充要条件是\underline{\hspace{1pc}$A_1A_2+B_1B_2+C_1C_2=0$\hspace{1pc}}
	\item 微分方程 $y''+2y'+5y=0$ 的通解为 $y=$\underline{\hspace{1pc}$C_1\ee^{-x}\sin(2x)+C_2\ee^{-x}\cos(2x)$\hspace{1pc}}
	\item 设 $z=\ee^{xy}+\cos\left(x^2+y\right)$, 则 $\frac{\partial z}{\partial y}=$\underline{\hspace{1pc}$x\ee^{xy}-\sin\left(x^2+y\right)$\hspace{1pc}}
\end{enumerate}
\subsubsection{大题}
\begin{enumerate}
	\item 求方程 $\frac{\dd z}{\dd x}=-z+4x$ 的通解
	\begin{solution}
		运用一阶线性非齐次微分方程公式, 得
		\begin{align*}
			z&=\ee^{-\int\dd x}\left( \int 4x\ee^{\int \dd x}\dd x+C\right) =\ee^{-x}\left( \int 4x\ee^{x}\dd x+C\right) \\
			&=\ee^{-x}\left( 4(x-1)\ee^{x}+C\right) =4(x-1)+C\ee^{-x}
		\end{align*}
	\end{solution}
	\item 求曲线 $2z+1=\ln(xy)+\ee^z$ 在点 $M_{0}(1,1,0)$ 处的切平面和法线方程
	\item 设由方程组 $\begin{cases}
	x+y+z=0\\
	x^2+y^2+z^2=1
	\end{cases}$
	确定了隐函数 $x=x(z),y=y(z)$ , 求 $\frac{\dd x}{\dd z},\frac{\dd y}{\dd z}$
	\begin{solution}
		对方程组 $\begin{cases}
		x+y+z=0\\
		x^2+y^2+z^2=1
		\end{cases}$ 两式求微分, 得
		\begin{equation*}
			\begin{cases}
			\dd x+\dd y+\dd z=0\\
			2x\dd x+2y\dd y+2z\dd z=0
			\end{cases}
		\end{equation*}
		解得
		\begin{equation*}
			\begin{cases}
			\frac{\dd x}{\dd z}=-\frac{x+2z}{2x+z}\\
			\frac{\dd y}{\dd z}=-\frac{y+2x}{2y+z}
			\end{cases}
		\end{equation*}
	\end{solution}
	\item 求方程 $y''+6y'+13y=\ee^t$ 的通解
	\begin{solution}
		方程 $y''+6y'+13y=\ee^t$ 对应的齐次方程 $y''+6y'+13y=0$ 的特征方程为 $r^2+6r+13=0$ , 解得 $r=-3\pm2\ii$ , 那么齐次方程的通解为 $C_1\ee^{-3t}\sin(2t)+C_2\ee^{-3t}\cos(2t)$
		
		设特解为 $a\ee^{t}$ , 代入方程 $y''+6y'+13y=\ee^t$ 后解得 $a=\frac{1}{20}$
		
		综上, 方程 $y''+6y'+13y=\ee^t$ 的通解为 $C_1\ee^{-3t}\sin(2t)+C_2\ee^{-3t}\cos(2t)+\frac{\ee^x}{20}$
	\end{solution}
	\item 设 $z=x^2y+\sin x+\varphi(xy+1)$ , 且 $\varphi(u)$ 具有一阶连续导数, 求 $\frac{\partial z}{\partial x},\frac{\partial z}{\partial y}$
	\begin{solution}
		$\frac{\partial z}{\partial x}=2xy+\cos x+y\varphi'(xy+1)$ , $\frac{\partial z}{\partial y}=x^2+x\varphi'(xy+1)$
	\end{solution}
\end{enumerate}



\section{高数(一)期终}
\subsection{2018-2019A15}
\subsection{2018-2019A15答案}




\section{高数(二)期中}
\subsection{2017-2018}
\subsection{2017-2018答案}
\subsection{2018-2019B10}




\section{高数(二)期终}
\subsection{2014-2015}
\subsection{2017-2018A}
\subsection{2017-2018A答案}
\subsection{2017-2018B}
\subsection{2017-2018B答案}

\section{额外的练习}


\chapter{线性代数试卷汇总}



\chapter{概率统计试卷汇总}



\chapter{复变函数试卷汇总}

\section{2018-2019A}
\subsubsection{选择题(每小题 $3$ 分, 共 $15$ 分)}
\begin{enumerate}
	\item $\frac{(\sqrt{3}-\ii)^{4}}{(1-\ii)^{8}}=$ (\hspace{1pc})
	\twoch{$-\frac{1}{2}+\frac{\sqrt{3}}{2}\ii$}{$-\frac{1}{8}\left(1+\sqrt{3}\ii\right)$}{$\frac{1}{8}\left(-1+\sqrt{3} \ii\right)$}{$-\frac{1}{2}-\frac{\sqrt{3}}{2} \ii$}
	
	\item 设 $f(z)=2 x^{3}+3 y^{3} \ii$ , 则 $f(z)$ (\hspace{1pc})
	\twoch{处处不可导}{仅在 $6x^2=9y^2$ 上可导, 处处不解析}{处处解析}{仅在 $(0,0)$ 点可导}
	
	\item 下列等式正确的是 (\hspace{1pc})
	\twoch{$\Ln \mathrm{i}=\left(2 k \uppi-\frac{\uppi}{2}\right) \ii, \ln \ii=\frac{\uppi}{2} \ii$}{$\Ln \ii=\left( 2k\uppi+\frac{\uppi}{2}\right)\ii,\ln\ii=-\frac{\uppi}{2}\ii $}{$\Ln \ii=\left(2 k \uppi+\frac{\uppi}{2}\right) \ii, \ln \ii=\frac{\uppi}{2} \ii$}{$\Ln \ii=\left(2 k \uppi-\frac{\uppi}{2}\right) \ii, \ln \ii=-\frac{\uppi}{2} \ii$}
	
	\item $z=0$ 是函数 $\frac{1-\cos z}{z-\sin z}$ 的 (\hspace{1pc})
	\fourch{本性奇点}{可去奇点}{二级极点}{一级极点}
	
	\item 设 $\mathrm{C}$ 为 $z=(1-\ii)t$ , $t$ 从 $1$ 到 $0$ 的一段, 则 $\int_{\mathrm{C}} \overline{z} \dd z=$ (\hspace{1pc})
	\fourch{$-1$}{$1$}{$-\ii$}{$\ii$}
\end{enumerate}

\subsubsection{填空题(每小题 $3$ 分, 共 $15$ 分)}
\begin{enumerate}
	\item 若 $z+|z|=2+\ii$ , 则 $z=$\underline{\hspace{8pc}}
	
	\item 若 $\mathrm{C}$ 为正向圆周 $|z|=\frac{1}{2}$ , 则 $\oint_{\mathrm{C}} \frac{1}{z-2} \dd z=$\underline{\hspace{8pc}}
	
	\item 若 $z=2-\uppi\ii$ , 则 $\ee^{z}=$\underline{\hspace{8pc}}
	
	\item 若 $f(z)=\cos z^2$ , 则 $f(z)$ 在 $z=0$ 处泰勒展开式中 $z^4$ 项的系数 $a_4=$\underline{\hspace{8pc}}
	
	\item 函数 $f(t)=\sin t$ 的拉普拉斯变换 $F(s)=$\underline{\hspace{8pc}}
\end{enumerate}

\subsubsection{计算题(70分)}
\begin{enumerate}
	\item 设 $u(x,y)=x-2xy$ 且 $f(0)=0$ , 求解析函数 $f(z)=u+\ii v$ . ( $10$ 分)
	
	\item 计算积分 $\oint_{\mathrm{C}}\frac{2\ee^x}{z^5}\dd z$ 的值, 其中 $\mathrm{C}$ 为正向圆周 $|z|=1$ . ( $7$ 分)
	
	\item 计算积分 $\oint_{\mathrm{C}}\frac{3z+5}{z^2-z}\dd z$ 的值, 其中 $\mathrm{C}$ 为正向圆周 $|z|=\frac{1}{2}$ . ( $7$ 分)
	
	\item 求函数 $\frac{1-\cos z}{z^3}$ 在有限奇点处的留数. ( $7$ 分)
	
	\item 求函数 $\frac{2z^2+1}{z^2+2z}$ 在有限奇点处的留数. ( $7$ 分)
	
	\item 将 $f(z)=\frac{z}{(z-2)(z-6)}$ 在 $2<|z|<6$ 内展开为洛朗级数. ( $10$ 分)
	
	\item 若函数 $f(z)=a y^{3}+b x^{2} y+\ii\left(x^{3}+c x y^{2}\right)$ 是复平面上的解析函数, 求 $a,b,c$ 的值. ( $12$ 分)
	
	\item 利用拉普拉斯变换解常微分方程初值问题: $\begin{cases}
	x''(t)+6x'(t)+9x(t)=\ee^{-3t}\\
	x(0)=0, x'(0)=0
	\end{cases}$ . ( $10$ 分)
\end{enumerate}



\section{2018-2019A答案}
\subsubsection{选择题(每小题 $3$ 分, 共 $15$ 分)}
\begin{enumerate}
	\item $\frac{(\sqrt{3}-\ii)^{4}}{(1-\ii)^{8}}=$ (\hspace{0.25pc}D\hspace{0.25pc})
	\twoch{$-\frac{1}{2}+\frac{\sqrt{3}}{2}\ii$}{$-\frac{1}{8}\left(1+\sqrt{3}\ii\right)$}{$\frac{1}{8}\left(-1+\sqrt{3} \ii\right)$}{$-\frac{1}{2}-\frac{\sqrt{3}}{2} \ii$}
	
	\item 设 $f(z)=2 x^{3}+3 y^{3} \ii$ , 则 $f(z)$ (\hspace{0.25pc}B\hspace{0.25pc})
	\twoch{处处不可导}{仅在 $6x^2=9y^2$ 上可导, 处处不解析}{处处解析}{仅在 $(0,0)$ 点可导}
	
	\item 下列等式正确的是 (\hspace{0.25pc}C\hspace{0.25pc})
	\twoch{$\Ln \mathrm{i}=\left(2 k \uppi-\frac{\uppi}{2}\right) \ii, \ln \ii=\frac{\uppi}{2} \ii$}{$\Ln \ii=\left( 2k\uppi+\frac{\uppi}{2}\right)\ii,\ln\ii=-\frac{\uppi}{2}\ii $}{$\Ln \ii=\left(2 k \uppi+\frac{\uppi}{2}\right) \ii, \ln \ii=\frac{\uppi}{2} \ii$}{$\Ln \ii=\left(2 k \uppi-\frac{\uppi}{2}\right) \ii, \ln \ii=-\frac{\uppi}{2} \ii$}
	
	\item $z=0$ 是函数 $\frac{1-\cos z}{z-\sin z}$ 的 (\hspace{0.25pc}D\hspace{0.25pc})
	\fourch{本性奇点}{可去奇点}{二级极点}{一级极点}
	
	\item 设 $\mathrm{C}$ 为 $z=(1-\ii)t$ , $t$ 从 $1$ 到 $0$ 的一段, 则 $\int_{\mathrm{C}} \overline{z} \dd z=$ (\hspace{0.25pc}A\hspace{0.25pc})
	\fourch{$-1$}{$1$}{$-\ii$}{$\ii$}
\end{enumerate}

\subsubsection{填空题(每小题 $3$ 分, 共 $15$ 分)}
\begin{enumerate}
	\item 若 $z+|z|=2+\ii$ , 则 $z=$\underline{\hspace{1pc}$\frac{3}{4}+\ii$\hspace{1pc}}
	
	\item 若 $\mathrm{C}$ 为正向圆周 $|z|=\frac{1}{2}$ , 则 $\oint_{\mathrm{C}} \frac{1}{z-2} \dd z=$\underline{\hspace{1pc}$0$\hspace{1pc}}
	
	\item 若 $z=2-\uppi\ii$ , 则 $\ee^{z}=$\underline{\hspace{1pc}$-\ee^2$\hspace{1pc}}
	
	\item 若 $f(z)=\cos z^2$ , 则 $f(z)$ 在 $z=0$ 处泰勒展开式中 $z^4$ 项的系数 $a_4=$\underline{\hspace{1pc}$-\frac{1}{2}$\hspace{1pc}}
	
	\item 函数 $f(t)=\sin t$ 的拉普拉斯变换 $F(s)=$\underline{\hspace{1pc}$\frac{1}{s^2+1}$\hspace{1pc}}
\end{enumerate}

\subsubsection{计算题(70分)}
\begin{enumerate}
	\item 设 $u(x,y)=x-2xy$ 且 $f(0)=0$ , 求解析函数 $f(z)=u+\ii v$ . ( $10$ 分)
	\begin{solution}
		解析函数的 $u,v$ 必定满足 $\mathrm{C}.-\mathrm{R}.$ 方程, 即
		\begin{equation*}
			\begin{cases}
			\frac{\partial u}{\partial x}=\frac{\partial v}{\partial y}\\
			\frac{\partial u}{\partial y}=-\frac{\partial v}{\partial x}
			\end{cases}
		\end{equation*}
		$\frac{\partial v}{\partial y}=\frac{\partial u}{\partial x}=1-2 y$ , $\frac{\partial v}{\partial y}$ 对 $y$ 积分得 $v=y-y^{2}+\varphi(x)$
		
		$\frac{\partial u}{\partial y}=-2 x=-\frac{\partial v}{\partial x}=-\varphi^{\prime}(x)$ , 可以得出 $\varphi(x)=x^{2}+C$
		
		由于 $f(0)=0$ , 因此 $C=0$ ,即 $f(z)=x-2 x y+\ii\left(y-y^{2}+x^{2}\right)$
	\end{solution}
	
	\item 计算积分 $\oint_{\mathrm{C}}\frac{2\ee^x}{z^5}\dd z$ 的值, 其中 $\mathrm{C}$ 为正向圆周 $|z|=1$ . ( $7$ 分)
	\begin{solution}
		根据高阶导数公式 $f^{(n)}(z_0)=\frac{n!}{2\uppi\ii}\oint_{\mathrm{C}}\frac{f(z)}{(z-z_0)^{n+1}}\dd z$ , 那么
		\begin{equation*}
			\oint_{\mathrm{C}} \frac{2 \ee^{z}}{(z-0)^{5}} \dd z=\frac{2 \uppi \ii}{4 !}\left.\left(2 \ee^{z}\right)^{(4)}\right|_{z=0}=\frac{\uppi \ii}{6}
		\end{equation*}
	\end{solution}
	
	\item 计算积分 $\oint_{\mathrm{C}}\frac{3z+5}{z^2-z}\dd z$ 的值, 其中 $\mathrm{C}$ 为正向圆周 $|z|=\frac{1}{2}$ . ( $7$ 分)
	
	\item 求函数 $\frac{1-\cos z}{z^3}$ 在有限奇点处的留数. ( $7$ 分)
	
	\item 求函数 $\frac{2z^2+1}{z^2+2z}$ 在有限奇点处的留数. ( $7$ 分)
	
	\item 将 $f(z)=\frac{z}{(z-2)(z-6)}$ 在 $2<|z|<6$ 内展开为洛朗级数. ( $10$ 分)
	
	\item 若函数 $f(z)=a y^{3}+b x^{2} y+\ii\left(x^{3}+c x y^{2}\right)$ 是复平面上的解析函数, 求 $a,b,c$ 的值. ( $12$ 分)
	
	\item 利用拉普拉斯变换解常微分方程初值问题: $\begin{cases}
	x''(t)+6x'(t)+9x(t)=\ee^{-3t}\\
	x(0)=0, x'(0)=0
	\end{cases}$ . ( $10$ 分)
\end{enumerate}


\end{document}
