\documentclass[cn,11pt,fancy,hide]{elegantbook}


\title{工科数学试卷汇总}
\subtitle{高数、线代、概率、复变}

\author{sikouhjw、xajzh}
\institute{临时组织起来的重排小组}
\date{\today}
\version{1.00}

\equote{“不论一个人的数学水平有多高,只要对数学拥有一颗真诚的心,他就在自己的心灵上得到了升华。”---SCIbird}

\logo{logo}
\cover{cover.jpg}
\newcommand{\ee}{\mathrm{e}}
\newcommand{\dd}{\mathrm{d}}
\renewcommand{\leq}{\leqslant}
%\everymath{\displaystyle}
\newcommand{\fourch}[4]{\\\begin{tabular}{*{4}{@{}p{3.5cm}}}(A)~#1 & (B)~#2 & (C)~#3 & (D)~#4\end{tabular}} % 一行
\newcommand{\twoch}[4]{\\\begin{tabular}{*{2}{@{}p{7cm}}}(A)~#1 & (B)~#2\end{tabular}\\\begin{tabular}{*{2}{@{}p{7cm}}}(C)~#3 & (D)~#4\end{tabular}}  %两行
\newcommand{\onech}[4]{\\(A)~#1 \\ (B)~#2 \\ (C)~#3 \\ (D)~#4}  % 四行
\begin{document}
\maketitle
\tableofcontents

% \thispagestyle{empty}

\mainmatter
\hypersetup{pageanchor=true}
\chapter{声明}
本汇总不得用于商业用途,最新版下载地址:\href{https://github.com/sikouhjw/LaTeX-learning-notes/tree/master/%E9%AB%98%E6%95%B0+%E7%BA%BF%E4%BB%A3+%E6%A6%82%E7%8E%87%E9%87%8D%E6%8E%92}{Github},不保证题目、答案的正确性,如有错误可通过QQ群\footnote{991832226}或者邮箱\footnote{489765924@qq.com}联系我们

\chapter{高等数学试卷汇总}

\section{高数(一)期中}

\subsection{2018-2019A7}
\subsubsection{选择题}
\begin{enumerate}[leftmargin=0pt]
	\item 微分方程 $(y')^3+3\sqrt{y''}+x^4y'''=\sin x$ 的阶数是(\hspace{0.5cm})
	\fourch{1}{4}{2}{3}
	\item 设 $f(x,y)=x-y-\sqrt{x^2+y^2}$ ,则 $f_{x}(3,4)=$(\hspace{0.5cm})
	\fourch{$\frac{3}{5}$}{$\frac{2}{5}$}{$-\frac{2}{5}$}{$\frac{1}{5}$}
	\item 微分方程 $y'=\frac{y}{x}$ 的一个特解是(\hspace{0.5cm})
	\fourch{$y=2x$}{$\ee^y=x$}{$y=x^2$}{$y=\ln x$}
	\item 若 $z=\ln\sqrt{1+x^2+y^2}$ ,则 $\left.\dd z\right|_{(1,1)}=$(\hspace{0.5cm})
	\fourch{$\frac{\dd x+\dd y}{3}$}{$\frac{\dd x+\dd y}{2}$}{$\frac{\dd x+\dd y}{1}$}{$3(\dd x+\dd y)$}
	\item 设直线 $L:\begin{cases}
	x+3y+2z+1=0\\
	2x-y-10z+3=0
	\end{cases}$ ,平面 $\eta:\ 4x-2y+z-2=0$ ,则(\hspace{0.5cm})
	\fourch{$L$ 在 $\eta$ 上}{$L$ 平行于 $\eta$}{$L$ 垂直于 $\eta$}{$L$ 与 $\eta$ 斜交}
	\item 方程 $y'+3xy=6xy$ 是(\hspace{0.5cm})
	\twoch{二阶微分方程}{非线性微分方程}{一阶线性非齐次微分方程}{可分离变量的微分方程}
	\item 曲面 $\frac{x^2}{9}-\frac{y^2}{4}+\frac{z^2}{4}=1$ 与平面 $x=y$ 的交线是(\hspace{0.5cm})
	\fourch{两条直线}{双曲线}{椭圆}{抛物线}
	\item 设 $z=\ee^{x^2y}$ ,则 $\frac{\partial^2z}{\partial x\partial y}=$(\hspace{0.5cm})
	\twoch{$2y(1+x^3)\ee^{x^2y}$}{$\ee^{x^2y}$}{$2x(1+x^2y)\ee^{x^2y}$}{$2x\ee^{x^2y}$}
	\item 下列结论正确的是(\hspace{0.5cm})
	\twoch{$\vec{a}\times\left(\vec{b}-\vec{c}\right)=\vec{a}\times\vec{b}-\vec{a}\times\vec{c}$}{若 $\vec{a}\times\vec{b}=\vec{a}\times\vec{c}$ 且 $\vec{a}\ne\vec{0}$ ,则 $\vec{b}=\vec{c}$}{$\vec{a}\times\vec{b}=\vec{b}\times\vec{a}$}{若 $\left|\vec{a}\right|=1,\left|\vec{b}\right|=1$ ,则 $\left|\vec{a}\times\vec{b}\right|=1$}
\end{enumerate}

\subsubsection{填空题}
\begin{enumerate}[leftmargin=0pt]
	\item 平面过点 $(2,0,0),(0,1,0),(0,0,0.5)$ ,则该平面的方程是\underline{\hspace{3cm}}
	\item 设 $y_1$ 是 $y''+p(x)y'+q(x)y=f(x)$ 的解,$y_2$ 是 $y''+p(x)y'+q(x)y=f(x)$ 的解,则 $y_1+y_2$ 是\underline{\hspace{3cm}}方程的解
	\item 设 $z=y\arctan x$ ,则 $\left.\mathrm{grad}\,z\right|_{(1,2)}=$\underline{\hspace{3cm}}
	\item 过点 $P(0,2,4)$ 且与两平面 $x+2z=1$ 和 $y-2z=2$ 平行的直线方程是\underline{\hspace{3cm}}
	\item 设 $f(x,y)=\arcsin\frac{y}{x}$ ,则 $f_y(1,0)=$\underline{\hspace{3cm}}
	\item $y=\ee^x$ 是微分方程 $y''+py'+6y=0$ 的一个特解,则 $p=$\underline{\hspace{3cm}}
	\item 已知平面 $\eta_1:\ A_1x+B_1y+C_1z+D_1=0$ 与平面 $\eta_2:\ A_2x+B_2y+C_2z+D_2=0$,则 $\eta_1\perp\eta_2$ 的充要条件是\underline{\hspace{3cm}}
	\item 微分方程 $y''+2y'+5y=0$ 的通解为 $y=$\underline{\hspace{3cm}}
	\item 设 $z=\ee^{xy}+\cos\left(x^2+y\right)$,则 $\frac{\partial z}{\partial y}=$\underline{\hspace{3cm}}
\end{enumerate}
\subsubsection{大题}
\begin{enumerate}[leftmargin=0pt]
	\item 求方程 $\frac{\dd z}{\dd x}=-z+4x$ 的通解
	\item 求曲线 $2z+1=\ln(xy)+\ee^z$ 在点 $M_{0}(1,1,0)$ 处的切平面和法线方程
	\item 设由方程组 $\begin{cases}
	x+y+z=0\\
	x^2+y^2+z^2=1
	\end{cases}$
	确定了隐函数 $x=x(z),y=y(z)$ ,求 $\frac{\dd x}{\dd z},\frac{\dd y}{\dd z}$
	\item 求方程 $y''+6y'+13y=\ee^t$ 的通解
	\item 设 $z=x^2y+\sin x+\varphi(xy+1)$ ,且 $\varphi(u)$ 具有一阶连续导数,求 $\frac{\partial z}{\partial x},\frac{\partial z}{\partial y}$
\end{enumerate}

\subsection{2018-2019A7答案}




\section{高数(一)期终}
\subsection{2018-2019A15}
\subsection{2018-2019A15答案}




\section{高数(二)期中}
\subsection{2017-2018}
\subsection{2017-2018答案}
\subsection{2018-2019B10}




\section{高数(二)期终}
\subsection{2014-2015}
\subsection{2017-2018A}
\subsection{2017-2018A答案}
\subsection{2017-2018B}
\subsection{2017-2018B答案}

\section{额外的练习}
\chapter{线性代数试卷汇总}

\chapter{概率统计试卷汇总}
\end{document}
